\chapter{Introduction}
\textit{Optical character recognition} (OCR) is the ability of the computer to extract textual information from image data, such as pixels\cite{schantz1982history}. This is useful because by introducing the data available on paper, we can use a computer to process, index and search the data much faster.  

OCR is a difficult problem, even when done on straight papers, without creases, that are scanned, because first we must identify letters on the page and distinguish them from tables, figures and other objects that might be there, and because there are many kinds of fonts that have to be recognized. The problem of OCR on photographed documents is even more difficult. The illumination can vary, the document might be curved, there might be a skewed perspective and so on. 

Information extraction is the process of taking raw, unstructured text and outputting information that is parsed and structured. This makes it easier to search and store the necessary data, because parts of the text that contain nothing useful can be discarded, while the rest is processed, brought to a standard form and is stored in a database. 

This thesis introduces an novel way of managing personal finances, by simplifying the input of expenses. Instead of requiring the user of the application to manually type in what expenses were made in each day, which can be tedious and time consuming, or to link his bank account to the application, which has potential security risks, expenses are introduces by taking a photo of a receipt. The application will then perform OCR on this receipt, extract the relevant information from the image and insert it into its database. The user can then see his expenses in an interactive dashboard and he can plan his financial future, helping him to save money. 

While there are many personal finance manangement applications, the novelty of ReceiptBudget is that it has an OCR engine that is tailored for receipts. By taking into consideration the constraints imposed by receipts, on both document layout and text font and spacing, we can improve performance compared to other OCR engines that are more general. The introduced approach is novel, since, to our knowledge, other papers for extracting information from receipts focus on improving either the image preprocessing step or on the parsing of text that was extracted using an off-the-shelf OCR engine.

There are many OCR engines, both commercial and open-source, but even though their results have greatly improved in the last couple of years, especially on large documents, on receipts their performance is lacking. This can be improved by building an engine that takes into consideration the constraints of the text layout in receipts. 

The rest of the thesis is structured as follows.

Chapter \ref{chap:statement} describes the problem statement and the motivation behind developing ReceiptBudget. Chapter \ref{chap:lit_rev} presents some of the related work in the field of character recognition using machine learning approaches and in the field of presenting information to users in the form of interactive dashboards. In chapter \ref{chap:application} we present the actual application, the way it was developed, and an experimental evaluation and comparison with other approaches. Finally, we provide conclusions and pointers towards future work. 

The OCR engine part of ReceiptBudget was also presented at the Sesiunea de Comunicari Stiintifice ale Studentilor. The application also won the first prize at the Imprezzio Software Contest 2013. 